\section{Introduction}

The Institute for Astronomy (IfA) is situated within the Royal Observatory on Blackford Hill, about two miles to the south of the city centre and about half a mile away from King's Buildings (KB), which is the main University science area.

The Royal Observatory Edinburgh (ROE) is an umbrella term embracing both the Institute (which is part of Edinburgh University's School of Physics and Astronomy) and the UK Astronomy Technology Centre (ATC), which is an establishment funded directly by STFC. It also contains an active Visitor Centre running extensive outreach programmes. The role of the ATC is technical support for British telescopes in Chile, Hawaii, La Palma and Australia, as well as development of front-line instrumentation for both ground and space-based telescopes. Both the IfA and the ATC are now members of the Scottish Universities Physics Alliance (which has the embarrassing acronym SUPA; \url{http://www.supa.ac.uk}). Most students will be part of the IfA exclusively but some may have a supervisor or some roll in the ATC as well. The staff and research facilities of the ATC provide valuable resources to complement those of the Institute and the students are made to feel very much part of the Observatory, with all the benefits that follow from that.

One disadvantage of the geographical location on Blackford Hill (apart from climbing up in snow in midwinter, or when it's really hot - which is never) is that there is little contact with other schools in the University. However the Institute is part of the School of Physics and Astronomy, which is located down the hill in the James Clerk Maxwell Building (JCMB) at the King's Buildings (KB) campus. This is profitable to postgraduate students in widening the range of available research seminars. The relatively recent creation of the Higgs Centre for Theoretical Physics has increased interaction between the two communities with a wide range of seminars which are often relevant to someone at the Observatory and always have an excellent array of biscuits. The School of Physics and Astronomy also encompasses the EPCC (Edinburgh Parallel Computing Centre) and NeSC (National e-Science Centre).

\section{Arrival}

New PhD students will usually arrive in September when they will find themselves with
a desk, computer and at least two supervisors. There is
also some initial administrative work that must be done, the two
most important pieces of which are Matriculation and Getting Paid.

\subsection{Getting Paid}

%{\bf Getting Paid}:

You should find out about how and when you get paid in your introduction day in the JCMB.
If you have any questions about this please don't hesitate to ask the relevant people
(see, for example, Liz Paterson liz.paterson@ed.ac.uk).

\subsection{Matriculation}

%{\bf Matriculation}:

Postgraduate students are required to matriculate at the start of
their period of study. This should be done online before your first day. All matriculated students will be given a University Card as proof of identity, which you can pick up during (or possibly before) your first week, from the university library in George Square.

\subsection{Talks to new students}

Several introductory talks will be given to you shortly after
your arrival. In addition, the current student representatives will meet with you when you arrive, to introduce themselves and talk about their roles in the department. At some point during the first few weeks of the semester, all PhD students attend the annual Slides Party (power-point slides only, sadly..). Each student then presents a single slide about themselves, giving a little bit of background information about hobbies, interests etc. The older students will also tell you about their areas of expertise, to make you more aware of who you should ask for specific help. The slides party is a really informal, fun evening, with no post-docs or faculty allowed until after the slides have been presented.

\section{Work}

\subsection{Supervisor meetings}

These are a very important and essential part of your PhD. Each
supervisor (and student) has different work routines, however as a
guide you should be meeting your supervisor once a week and certainly
at least once a fortnight.  If possible schedule a weekly meeting time
and place, even if you think that the task you are currently working on
will take longer than this it is important to keep in touch with your supervisor and it also keeps
you motivated.  In many cases just meeting for 5 minutes to give an
update on what you have been up to can lead to a discussion that will
solve some of your problems.

All students in the Institute have two supervisors, occasionally
including one from the ATC staff. In many cases the student
works almost exclusively with one of them,
on the day--to--day level, while the other supervisor keeps a watching
brief, following the progress made by the student.

In the great majority of cases the relationshi between student and supervisor proves to be
amicable and profitable,  but in a very few cases
it does not and so it is desirable for there to be procedures in place
to handle such situations when they arise and, hopefully, to nip them
in the bud before there is any danger of their becoming too serious.

Depending on the circumstances, PhD students with any sort of problem
can speak to the Postgraduate Course Organiser,
The Postgraduate Rep, the Head of Institute or the School Director of Graduate Studies, all of whom
will be happy to listen and act as an intermediary if required.

In addition to meetings with supervisors, you will meet roughly annually with a member of staff from another part of the physics department. The purpose of these meetings is to provide the opportunity to talk about any issues you may be having, PhD-related or otherwise. These meetings are compulsory, but informal, and can be useful if you need someone to talk to outside of the astronomy department. 

\subsection{Reading groups}

There are many weekly reading group meeting in which new papers are read and dissected or reviews gone through. This is a useful way to interact with more members of the department, and learn about science beyond your specific subject area. Check with your supervisors to see which one(s) might be good to attend and ask the organisers for details of how to join the meetings. Current reading groups (non-exhaustive):
\begin{itemize}
    \item Astro-ML (Machine learning in astronomy); 11:00 first Monday of each month. Contact: Sarah Appleby
    \item Local Universe (Local/Milky Way); 12:30 every Monday. Contact: Mike Petersen \& Pete Kuzma
    \item Exoplanets (Exoplanets and solar system astronomy); 11:00 every Tuesday. Contact: Sophie Dubber
    \item High-z (Extragalactic astronomy); 12:00 every Tuesday. Contact: Adam Carnall
    \item LSS (Large scale structure and cosmology); 14:00 every Tuesday. Contact: Daniele Sorini
    \item Lensing lunch (Weak and strong gravitational lensing); 13:00 every Wednesday. Contact: Qianli Xia
    \item Theory lunch (Astrophysics theory discussions); 13:00 every Wednesday. Contact: Avery Meiksin
\end{itemize}

{\bf 2020 note: reading groups now take place remotely over Zoom/Teams.}

\subsection{arXiv}

All new astronomy papers are released to the public arXiv before being published.  You should set yourself up to get emailed every new paper released in your field each day (info here: \url{https://arxiv.org/help/subscribe}), and get into the habit of
reading through at least the abstract of each new paper each day to check that you don't miss something important in the field.

\subsection{Seminars etc.}

For most of the weeks of the year (the exception being the summer period) there is an hour long colloquium lecture, usually given by a visiting academic. Normally, the speaker will meet with the PhD students for 20 minutes ahead of the colloquium in order to provide a simplified guide to the topic and give the students an opportunity to ask questions in an informal environment.  The colloquium itself is supposed to start with a basic introduction to the topic before getting into the details of a person's specific research into a specific field. The seminar speaker is usually taken for dinner after the seminar and students are able to come along at a heavily subsidised cost. This is a great way to get to know the faces of astrophysics and to simultaneously eat great food for cheap.

Each Friday there is a `Stobie Talk' in the Stobie room which is a short (10 minute) talk given by someone in the department and followed by doughnuts. Students are expected to give Stobie talks during their time at the ROE.

There are also sporadic `coffee talks' given by other visitors which happen over coffee and are held in the canteen.

The program for all of these events can be found at \url{https://ifa.roe.ac.uk/seminars-events} and
a helpful calendar extension can be found here as well.

There are also seminars held in the JCMB which can be relevant. 
Details of these will be emailed around the department as and when.

{\bf 2020 note:  weekly seminars and Stobie talks now take place on the `Seminars' and `Stobie' channels on Teams for the foreseeable future. The same format for these talks still applies (unfortunately the department won't supply Stobie doughnuts to your flat). Whether using Zoom, Teams, etc, it is considered good etiquette to keep your microphones muted while others are presenting to limit background noise, and to use the 'raise hand' feature if you have a question.}

\section{Lifestyle}

\subsection{Hours and holidays}

It can be difficult to know how many hours you should be working,
especially as the time your supervisor thinks a task should take is
often an underestimation.  Everybody works differently, but in general if you come into work every weekday for a reasonable number of relatively productive hours then you will do fine. However, 
there will undoubtedly be times where a major push is required to meet an important deadline, and it is a big mistake to regard a PhD as a 9-to-5 job. Certainly to begin with, as long as your supervisor is happy with your effort levels then you are doing enough. It is also important to remember that the post-doctoral job market is very competitive, and only those who have worked hard and effectively during their PhD are likely to be able to pursue a career in astrophysics.

Some students prefer to work from home, however this often results in missing out on important departmental events including reading groups and seminars and is therefore discouraged. The Observatory is also the best place to find quick help for problems, and discussions with fellow PhD students can be useful or at least mildly interesting.  Also, especially when first starting your PhD, working exclusively from home means missing out on socialising with your fellow PhD students (particularly during the many daily coffee breaks). Try not to underestimate how important it is to get to know the other students - they will inevitably end up being some of your closest friends in Edinburgh (whether you like it or not). 

{\bf 2020 note: obviously, this advice to not work from home no longer applies and the current guidance is that if you can work effectively from home then you are encouraged to do so.}

Remember to take your holidays, but discuss carefully with your supervisor when best to take these given other unmovable deadlines (which apply to telescope proposals and conference presentations etc). There is no strict limit imposed for holiday time at the IfA, however a sensible number of weeks to take off a year is a maximum of 6 in total. Usually, using the official holiday booking system is unnecessary - the most important thing is to get the OK from your supervisor first. 

\subsection{Mailing lists}

The most common mailing list used in the ROE is the `all-roe@lists.roe.ac.uk' mailing list, into which
everyones work email address is automatically entered. Sending to
the all-roe list sends your email to everyone on site, so be careful how you use it! Whilst
it is used primarily for work related issues, it is common that emails will be sent for other
 reasons (for example, help with finding flat mates), 
and it is a great way to access the large quantity of experience within
the ROE, not necessarily related directly to astronomy or the running of the site. If the email is
not directly related to the runnings of the ROE or work, the etiquette is usually to mark
these emails as `SPAM' in the email subject line.

A similar mailing list is `all-ifa@lists.roe.ac.uk' which sends your message just to the staff, post-docs
and students of the IfA. You are automatically added to this mailing list when you start.

To email to all astronomy PhD students use `pg-as@ph.ed.ac.uk', or for all physics postgrads use `pgstudents@ph.ed.ac.uk'. 

\subsection{Canteen}

The ROE is a little bit in the middle of nowhere, so it is convenient that we have a
canteen on site where it is possible to buy hot and cold food and drink. The meals are
subsidised, so are cheaper than you may expect. 
Many people bring in their own food, heat it up in one of the microwaves and eat
it in the canteen with everyone else. There is also a cafe on the top floor of the Higgs building, that serves coffee, cake and other snacks.

Alternatively, there are a few canteens in KB which are convenient if you happen to
be in the area (for example, to teach).

{\bf 2020 note: the canteen at the ROE is currently closed; for now you should remember to bring your own food in and avoid sharing food with others.}

\subsection{Pubs}

It is sometimes arranged that on Friday (or any other day given the slightest excuse),
people from the IfA go to the pub after work. The particular location will vary from excursion to
excursion but, for the most part, will be somewhere in Morningside, Newington or nearby areas
of Edinburgh. Some of the favourite PhD student pubs include the Old Bell and the Southern, both in Newington, and the Golf Tavern in Bruntsfield. Usually, the idea of a pub trip will be raised on the PhD Facebook page, so keep an eye on this! 

{\bf 2020 note: no doubt pub trips will recommence in the near future, but will probably be limited to small groups of people.}

\subsection{Teaching}

PhD students can become teaching assistants (TAs) and teach the
undergrads. What this involves varies wildly from course to course,
there are workshops, labs, and computing course.
Teaching is great fun and good experience/revision for
you, as well as providing some extra cash.  Most people teach one or
two courses a term, but bear in mind the number of hours taken out of
the day as well as the preparation/marking hours.  At the beginning of
the academic year an e-mail is sent round with details of how to sign
up for teaching posts along with details about how to get paid.  You
will need to fill out time sheets each month (note the final day for
submission to be paid that month), these are then confirmed by the
head of each course and processed by the physics accounts office. You will also need to undergo some basic training, including an in-person course at the start of the semester, and some online courses (e.g Data protection training).

Details about how to fill in time sheets as well as course codes etc. should be given to you at your introduction to teaching session.

{\bf 2020 note: with social distancing, the undergraduate courses will now take place as a hybrid mix of in person and online classes, which in practice means there are more teaching hours available for TAs to pick up.}

\subsection{Reps}

There are several rep positions held by the PhD students that should
be distributed within each year.  These help everything run smoothly
in the PhD community and help you find the right person to talk to.

\begin{enumerate}

\item Postgraduate Forum reps: Freddie Phipps: 	phipps@roe.ac.uk, Dylan Robson: dylanr@roe.ac.uk, Fran Lane: flane@roe.ac.uk:
\begin{itemize}
    \item attend one or two meetings a year with the head of postgraduate studies (for the School of Physics and Astronomy, not Catherine)
    \item organise joint physics and astronomy events (such as the Firbush trip and the Christmas ceilidh)
    \item voice concerns the astronomy students have about general school matters
    \item have input in things such as annual review system and SUPA courses
\end{itemize}
\item Student reps: Harry Rendell-Bhatti: hrb@roe.ac.uk and Ryan Begley: rbeg@roe.ac.uk = slides party, secret santa, updating the handbook \& general issues or problems you might have!
\item Computer rep: Ryan Begley: rbeg@roe.ac.uk = help with computing problems, reporting issues at the annual meeting.
\item Library rep: Massi Hamadouche: mham@roe.ac.uk = helps with library related issues
\item Teaching rep: Robert McGibbon: rmcg@roe.ac.uk = helps with teaching related issues
\item Doughnut rep: Massi Hamadouche: mham@roe.ac.uk = organise the rota of people to collect doughnuts for Stobie Coffee (bought by a different 2nd year each week)


\end{enumerate}


\subsection{Firbush}

The School of Physics also organises an annual trip to Firbush Outdoor Centre, the University's outdoor centre on Loch Tay. This trip is heavily subsidised by the department and allows astronomers a chance to meet some of the other members of the Physics Department in an informal setting. You get the chance to learn any number of new outdoors skills or just relax with a beer!

\subsection{The Library}

The Royal Observatory has a very substantial working library of
astronomical books and periodicals and, in the Crawford Collection,
arguably the world's finest collection of rare and historical books on
astronomy and related topics. Most of the books, excepting those in
the Crawford Collection, may be borrowed from the Library, subject to
the certain restrictions (for example, some are reference only).

One of the library staff will give new research students a tour of the
library, including the Crawford Collection, as well as an
introduction to the indexing and borrowing systems used.

Photocopying of library material may be done, subject to copyright
restrictions.
 
\subsection{The Lodge and Security}

The Lodge contains the Observatory's reception area, as well as
pigeonholes for all staff and students. The Lodge is
also one of the points on the internal University mail system, which
connects departments and University offices throughout the city. You can get both letters and parcels delivered to the lodge, and will be informed by email when packages arrive so that you can pick them up.

All visitors to the Observatory, except those going to the Visitor
Centre, must be signed in at the Lodge.  You
can access most areas of the site using a card which will be issued 
to you when you
arrive and must be returned when you finish your PhD. When the glass
doors to the 1967 building are locked you must use the
library doors to enter and exit the IfA. 

{\bf 2020 note: with social distancing the current guidelines on entering the site have changed - see \href{https://tinyurl.com/yyvxhjcx}{this document} circulated by Philip on 7/9/2020.}

\subsection{Visitor Centre}

The ROE Visitor Centre has a permanent exhibition open to school
visits only throughout the year and also organises special events, such as Open Evenings, a series of lectures over the winter months, evening visits from organised groups and `Popular Observing', which opens the remaining working telescopes in the Observatory to public use over the
winter. The Visitor Centre has a permanent staff, but students are
employed to run the `Popular
Observing' and to address groups of visitors in the
evenings. Students are strongly encouraged by the Institute to help
the Visitor Centre with groups. The benefit acts in two ways: talking
in public is something all research scientists have to do, and this is
good experience; secondly, society funds astronomy, so we have some
obligation to give something in return. Giving the public the chance to
hear from and talk to active researchers is a good way to do this, and
to further the public understanding of science.

\subsection{Stationery}

A supply of stationery for use by University staff and students is
available from the stationery cupboard in the IfA Teaching Office,
This should be raided for pens, paper etc. If
you find certain supplies are running low, please let someone in the Teaching Office know and they can order more.

\section{Travel}

A perk of being an astronomer is the travel opportunities.  Each year
there will be one or two (or even more depending on the field)
conferences where your supervisors, the authors of the papers that you
have been reading and even eventually \emph{you} will be
presenting their work.  As part of your PhD funding there is money
set-aside for travel.  Not all students travel to the same degree or
spend the same amount of money each year, so although there may be a
set amount of money for each student per year, this isn't a fixed amount.  If your conference
is expensive then you can probably still go, but bear in mind that in
later years you may be denied an expensive conference so that for
each student the money spent averages out.

There are also summer (or winter, or whatever) schools that one can attend
where formal lectures are given on specific topics. These are particularly
useful in your first couple of years when you are still finding your feet.
Make sure to look for these as your supervisors may not know about them.

There is also a separate pot of money for field trips that traditionally has been used for students to go observing, but that can now be justified by gathering data in other ways (e.g. simulation data, attending summer schools, etc). If your trip involves bringing back some new data for use in your project then you may be able to use this money. The procedure is still a little uncertain as it's not clear what the IfA can and can't use it for, but if you suspect that you could benefit from it then check with Philip Best.

Procedure for travel is as follows:
Find the conference, your supervisor may tell you about one or you
might find it yourself. The first thing you should do is speak with your supervisor who can advise as to whether or not the travel is suitable and how to go about getting it approved. Don't be scared to suggest relevant conferences to your supervisor, they may not have noticed it, and if you don't ask then you won't get to go.  If approved, register for the conference (note the deadline for this might be many months before the conference!) and
book your hotel/travel. Claim back the money for these via the university e-Expenses system,
which you should already have registered for (see below).  Note it takes a long
time to get an e-Expenses account so do this early. If you're spending over \pounds300 on your hotel and travel, you should either book it through the university travel agent (Key Travel) or get 3 different quotes before you book as evidence that you haven't gone overboard with your spending. You can book through Key Travel directly, or send an email to admin-general@ph.ed.ac.uk where one of the administrators will help you. In your email, you should include:
\begin{itemize}
    \item your name as it appears on your passport
    \item your passport number
    \item any frequent flyer membership numbers
    \item whether you require hold luggage
    \item reason for travel
    \item the travel/accomodation required, with as much detail as possible, i.e. if you know flight numbers/times then include them
    \item where the costs should be charged to (grant name, cost centre, account code and job centre if known, but don't worry if you don't have this information)
    \item who authorises the travel, i.e. who is your supervisor. You shoudl cc your supervisor into the email.
\end{itemize}
In general it is more straightforward to do this through the school admin team as you don't have to claim the money back yourself. They can also help you get (free) travel insurance through the university.

{\bf 2020 note: although travel has been restricted recently, many conferences have been taking place online. It's worth checking whether there are any online meetings coming up that you can take part in. }

\subsection{NAM}

The National Astronomy Meeting (NAM) is an annual event in the
UK. This takes place around Easter or early Summer each year and lasts for a few days, students and staff from
institutions all over the UK attend. The IfA encourages
students to go (if relevant) and costs will be covered by the Institute.

\subsection{SUPA/Cormack Astronomy Meeting}

In 1942, the Royal Society of
Edinburgh (RSE) was made the residuary legatee of the estate of Robert
Cormack, a local businessman, the society being `directed to
administer this bequest for the purpose of promoting astronomical
knowledge and research in Scotland'. Under the terms of the bequest,
the society makes a number of research fellowships and grants, as well
as holding an annual meeting each November. It is now also part funded by the Scottish Universities Physics Alliance (SUPA). Most years the meeting
is held in Edinburgh, in the Society's rooms on George Street. In the
case it is held outside Edinburgh the
Institute will pay travel expenses. Members of the astronomy groups at
Edinburgh, Glasgow and St. Andrews Universities are invited to this
meeting, to give talks on their work, listen to other such talks, present posters and
have a very good tea afterwards.

\subsection{e-Expenses}

Money for travel expenses can be reclaimed using the e-Expenses system run by the University.
The essentials about what
can be claimed and how to claim it can be found in \url{https://www.wiki.ed.ac.uk/display/Finance/eExpenses}.
You need to register for an account and this can take a month or months to sort out (inexplicably) so it is best to
do this early on in your PhD. Louise Alexander (Finance Assistant) in the Teaching Office (C17) can help you get this set up.

An important point to note is that the `description' field must be filled with the location and purpose of the claim
(\emph{e.g. Heidelberg - Dark Universe Conference}) for every entry.
Once the claim has been submitted online it must be printed off,
receipts attached and given to Louise. If she isn't there, you can leave claims in the tray in the Teaching Office.

TODO: check that Louise is still the right person to contact about this and check the remote procedures for claiming expenses.

\section{Yearly reports and assessments}

\subsection{1st year}

You are required to produce a First Year Report, this is probably the most substantial
document you will have to produce before your thesis. You will be
e-mailed details of this nearer the time, but it mainly consists of a literature review on your topic and a description of your progress so far (\(\sim\)15-20 pages), and a short summary for a general physics audience (\(\sim\)~5 pages). This typically takes around 2-3 weeks to write, depending on the individual. It's worth taking the time at this stage to read through the literature properly and clarify anything you don't understand with your supervisor. From the start of your project you should keep track of the papers you've read and the main references for your field - it will make the literature review go a lot quicker. What you produce will be read and assessed by two members of staff 
(not your supervisors) and you will have a mock Viva (around 1 hour long) over the summer.
Reports are written by your assessors and also your supervisors, which
are then given to you.

You are also expected give a Stobie talk to the whole department at some point during your first year.
This should be a short talk about what research you
have been up to.  Don't worry too much about this, it is supposed to
be informal and is mainly a way for the staff to get to know your
name.

\subsubsection{Reading Group}

The first-year reading group is attended by each student during the first year at Edinburgh.  The group meets thirteen times between October and March.  Ahead of each meeting some reading and an exercise is set by a member of academic staff and answers submitted by each student.  The member of staff will then discuss the topic with the students in an hour-long meeting.  Students are expected to participate actively in these discussions. The expectations of how much work this involves varies between academics, but usually there are around \(\sim\)10 pages of reading and a few questions. Make sure you set enough time aside each week to do this. Also, check with older students before you go ahead and print out the reading as most of them already have copies that they can give you.

\subsubsection{SUPA Courses}

As part of the first year of your PhD you are expected to do some academic courses (60 credits overall, split into 40 credits of physics/astronomy topics and 20 credits of transferable skills). The
reading group counts for 20 credits towards this but you will still have to do at least one other
academic course. The lectures and assessment for this vary from course to course and full
information on this can be found from SUPA (who organise all the graduate physics courses in Scotland)
\url{http://www.supa.ac.uk/Graduate_School/Graduate_School.htm}.
You are also able to attend any Undergraduate/Masters courses that the University offers
and may be able to get credit for these -- check with the course.
Your supervisors may also recommend that you attend specific courses. 

\subsubsection{Transferable Skills}

In recent years, the College of Science and Engineering
has been developing a major programme of transferable skills training
for postgraduates, reflecting the growing importance of these skills both
for successful completion of a PhD and for future career prospects.
Courses include such things as Effective Presentations, Scientific Paper
Production, Tutoring and Demonstrating, Team Development etc.\ as well as a
 number of computing courses. These courses take place throughout the year, 
and are advertised well in advance, but details can also be found on the
Web at: http://www.ed.ac.uk/schools-departments/institute-academic-development/postgraduate/doctoral/\\ courses/science-engineering

These courses can count towards your transferable skills training requirement. Traditionally students during their PhD tend to be quite negative about theses courses, but students surveyed after completing their PhD identified their failure to attend more such courses during their PhD as their biggest regret.

\subsection{2nd year}

You are required to produce a Second Year Report; a short summary of what you have achieved in
the second year and a time line for thesis completion. This is assessed by a 
short viva. You are also expected to produce a poster, that will be displayed in a large poster display competition.

\subsection{3rd year}

You are required to produce a very short third year report which includes a thesis plan, and also to give a 30 minute seminar about your work.

\subsection{4th year}

You are required to produce your thesis (obviously).

\section{Computing}

\subsection*{Caveats}

The majority of this section has been written by some serious computer geeks! If you don't understand any of the terminology in it then fear not, you will pick up what you need to know quickly from the people around you. Usually the fastest way of solving your problems is to ask one of the more experienced PhD students because they will almost certainly have dealt with your very problem many times over. There will almost always be someone to talk to who is more experienced in using the coding language or software that you are trying to use and they will be flattered and eager to help if you ask them questions.

If you have not used much Linux before coming here or programmed much before then it can be daunting, however these are skills that you will quickly pick up and that are much less intimidating than they first appear. Don't be embarressed if you don't know much about computers - lots of other people are in the same boat.

\subsection*{Mac users}
You should read all these notes, but if you plan on primarily using your own MacBook there are some extra tips starting in Section \ref{macusershelp}.

\subsection{Introduction}
This section aims to provide a (brief!) introduction to the IfA's Debian/GNU Linux computing system, and to the facilities available to staff and students for backups, accessing licensed software, sharing your data with collaborators at different institutions, accessing your work machine from outside the site and using your laptop on-site. This guide is \textit{not} an exhaustive tutorial in the intricacies of the operating system, nor will it provide much detail on individual programming languages except where ROE-specific quirks are relevant - the Internet is awash with such information, which is easily Google-able, and comes from much more qualified sources than any of the current PhD students! We do hope, however, that the following few shreds of wisdom, collated through thousands of student-hours of enquiry, trial-and-error, and perseverance in the face of despair will smooth your introduction to ROE life.

\subsection{Getting started}
As part of your induction process, you should by now have been given a user name and password for access to the IfA Debian/GNU Linux computers, and been allocated a desk and workstation. We'll now cover a few things you might like to look into as you begin to find your way around.

\subsection{Help desk}

The IT helpdesk is one of the most useful things you will come across as part of your PhD. 
It allows you to submit a `ticket' for any IT related
issues you may be having and get a very speedy response so that these problems do not eat into your research time. The helpdesk is typically the only way round most computing problems that require administrator rights to rectify. 

The IT helpdesk system can be accessed by emailing `roeitsupport@stfc.ac.uk' with your issue, including your ROE username and desktop name.

For estates issues use the online ticketing system, which can be accessed by going to the main ROE web page, and clicking the link `Intranet' near the bottom of the right-hand menu and then choosing `ROE Helpdesk System'.

Do not be afraid to use the helpdesk system! It's often best to initially check with an elder PhD student in person to see if your problem can be solved quickly but if not then do not hesitate to send a ticket. 


\subsection{e-mail}

You should be given an ROE e-mail account very soon after you arrive.

\subsection{Window manager}
Depending on your preferences, the first thing you might like to do is change your window manager. By default, the IfA machines typically use the {\sc MATE} window manager when you log in, which is quite far removed from both Microsoft Windows and Mac OS in its behaviour and appearance. Many people prefer the look and feel of {\sc gnome}. You can freely switch between these two window managers from the login screen -- enter your user name when prompted, and when you're asked for your password, a few buttons will appear at the bottom of the screen allowing you to change the language/keyboard layout and also to change the window manager between {\sc fvwm}, {\sc gnome} and ``{\sc system default}''. Choose whichever window manager you most like the look and feel of!

\subsection{Your home space}
The reason it's possible to log in to your IfA profile from any machine on site is that your profile is not held on your own particular workstation's hard drives, but on a centralised server. Along with your user profile, a home directory (/home/[your user name]) is also stored on this server. The main purpose of this directory is to store ancillary files associated with your user profile (such as your .login file -- more on which later) and temporary files from your web browser. Disk space on {\sc moray} is strictly quota'd -- to find out your disk usage/allowance, simply open a terminal window (Applications $\rightarrow$\ Accessories $\rightarrow$\ Terminal if using {\sc gnome}), type ``quota'' and hit enter. You're limited both in terms of disk space used, and also in terms of the number of individual files stored in your home directory. The most likely culprit for filling up your quota is the morass of temporary files stored by your web browser - please try to clear your browser cache at regular intervals if you're having trouble writing files to your home directory! Another big culprit
of wasting number of files is the `.thumbnails' folder. There are many hidden folders on your home space so remember to
check all of these when looking for what is clogging up your quota. Ask older students for help clearing your cache - there are some useful scripts that have been shared between students that do this quickly.

\subsection{Getting more storage space}
By default, you will have access to storage space in your remote home space on {\sc moray}. However, you should also have access to your workstation's internal hard drives (/disk*/) for storing larger data sets.  If you do not have space on your desktop's internal drives, you will need to submit a helpdesk ticket requesting this and specifying the name of your machine (which should be printed in a fairly obvious place on a sticker on your PC tower). You should do this as soon as you are given a computer, as this process can sometimes take quite a long time and delay you starting your project.


\subsection{Logging in to other machines on-site}
For various reasons, you might need access to one machine on-site while sitting at another. The protocol used to facilitate this is Secure SHell {\sc ssh}. For example, to log in to the ROE undergraduate server {\sc blackford}, bring up a terminal and type

{\bf ssh [username]@blackford.roe.ac.uk}.

This will enable you to run terminal-based (text-only) programs on {\sc blackford} as if you were sitting in front of it. To open a program on {\sc blackford} that has a graphical interface, or needs to open a separate window to function, use the ``-X'' option which means ``forward any X11 windows across the connection'', e.g.

{\bf ssh [username]@blackford.roe.ac.uk -X}

and then run your program as normal.

\subsection{Setting automated backups}

Now that you have configured your computer to your liking, gained access to its local hard disks and learnt how to access other machines on-site remotely, you should now seriously consider setting up automated backups to keep your data safe in the event of a disaster. All IfA members are entitled to a backup directory on the backup server. Eric Tittley has written the definitive guide to custom backups, which can be found at 

\url{http://www.roe.ac.uk/~ert/backup_schedule/}

The important steps, which are discussed in more detail on Eric's page, are to

\begin{enumerate}
\item Submit a helpdesk ticket requesting a directory on the backup server
\item Choose a time from the available time slots at which the daily backup can take place (this is important in order to smooth-out data flow across the network and prevent a dozen people all throwing hundreds of GB of data at the backup server at the same time!)
\item Enable password-less logins -- thus far, when SSH'ing into another machine you've had to supply your password each time. You'll want to set up password-less logins so that the backup server can talk to your computer without you having to enter a password
\item Determine the {\bf rsync} command you'll need to mirror a directory on your machine 
to your backup directory on the backup server, and learn to use {\bf cron} to automatically execute that command at the given time each day.
\end{enumerate}

The instructions may seem overwhelming to begin with, but they are logical, and the process need only be done once after which you'll enjoy automated back ups (subject to available disk space...) for the duration of your PhD.

\subsection{Running programs at log-in and establishing aliases -- the .login file}
One of the most useful little files stored in your home directory is the .login file. This file can be used to store aliases and shorthands for long commands that you might not want to type in full each time you need them. It is read each time you log in, and all commands that are stored in it are executed in order.

For instance, many people like their computer to say \textit{Sarah and Sophie are the best!} to them each time they log in and open a terminal window. You can do this manually if you like, by bringing up a terminal window and typing 

{\bf echo `Sarah and Sophie are the best!'}

Alternatively, you can open .login with a text editor (e.g. {\sc gedit}) and set up an \textit{alias} to cut down on the typing needed to receive the message. In your .login file, you could enter the line

{\bf alias test `Sarah and Sophie are the best!'}

You can then save the file, open a new terminal window and simply type {\bf test} to be greeted with the same message. The alias simply binds a long command (echo `Sarah and Sophie are the best!') to a shorthand, meaning that you need only type the shorthand from the terminal in order for the long command to be executed.

Alternatively, you can have the command run completely automatically and without any prompting when you open a terminal window by editing .login, removing the line you just added and replacing it with {\bf echo `Sarah and Sophie are the best!'}. You should now be greeted each time you open a terminal window.

Setting up aliases is useful if you need to run something for which the terminal command is long/complicated and don't want to have to type the whole thing in by hand each time. This is especially useful if you want to have access to custom-installed software, the location of which you will need to feed in to the terminal for it to work.

\subsection{Software sources}
The IfA machines can compile and run programs you've written in most of the major software languages -- Java, Python, Perl, IDL, Matlab, Fortran, C and its derivatives, etc. -- and you will be able to access a few common astronomy software packages (e.g. the Starlink package, GILDAS, IRAF, AIPS, CASA and a whole host of other telescope-specific data reduction packages). Some of these packages aren't available by default to all users (you'll need to look up the relevant pages on the ROE Intranet and edit your .login file to tell your machine where to look for them), and for certain other packages, the IT-maintained version is months or even years out of date and it's best to install your own version anyway. 

If you need a specific software package and don't know if it's available or not, your first port of call should be to submit a helpdesk ticket. If you need software and know that there \textit{isn't} a suitable version installed (and IT won't install it for whatever reason), then you can install certain packages on to your own machine without IT privileges. For instance, to install the latest version of the radio astronomy package CASA, you need only download the *tar.gz file, saving it to a directory of your choice (e.g. /disk1/[username]/CASA). You may then unpack it ({\bf tar -zxf casapy-34.0.19988-002-64b.tar.gz}) and establish an alias in your .login file:

{\bf alias casapy `/disk1/[username]/bin/CASA/casapy-34.0.19988-002-64b/casapy'}

Which will then allow you to open CASA v34.0.19988-002 simply by typing {\sc casapy} at the terminal. Similar steps can be taken for the installation of other programs and packages.

For certain proprietary software packages (e.g. IDL, Mathematica), a license is needed on any machine before the software can be run. License files, where appropriate, are automatically available on ROE workstations. Where necessary, the license can often be extended to cover personal laptops -- you would need to request this service from IT via the helpdesk system.

\subsection{Remote access to your ROE computer}
In principle, IfA machines may be accessed remotely from anywhere in the world via SSH. The only snag is that for the most part, IfA workstations aren't visible to the outside world, and in order to gain access to your work machine from elsewhere (e.g. from home) you must ``trick'' the IfA network into thinking you're on-site. There are two main ways of doing this.

\begin{enumerate}
\item {\bf SSH'ing through a trusted gateway} (archaic): the ``old fashioned'' way of doing this would be to SSH from your home machine in to a ``friendly'' University computer that \textit{is} visible to the outside world, and which is allowed to act as an intermediary to the ROE computers. Former UoE undergrads should already have access to the Physics server {\sc manuel.ph.ed.ac.uk}, and postgraduates should be able to request access to it by emailing the appropriate IT person at JCMB and making a trip down to JCMB to collect your login details. {\sc manuel} can be SSH'd into from anywhere ({\bf ssh [username]@manuel.ph.ed.ac.uk -X}). {\sc manuel} can then talk to the ROE machine {\sc katrine} ({\bf ssh [username]@katrine.roe.ac.uk}), and {\sc katrine} can then talk to any workstation on-site.
\item {\bf VPN} (the recommended alternative): rather than having to establish a chain of SSH's going ``your home machine'' $\rightarrow$\ {\sc manuel} $\rightarrow$\ {\sc katrine} $\rightarrow$\ ``your IfA machine'', it is now possible to use a VPN client installed on your home machine to ``shake hands'' with the ROE network and hence let you access any ROE machine directly from home without the middle-man. To make use of this facility, you'll need to submit a helpdesk ticket requesting VPN access. Horst will then send you the digital certificate and summon you to collect yet another user name and password combination, which you will be expected to memorise before eating the piece of paper on which it is printed. Then, from home and with the VPN running you'll be able to {\bf ssh [username]@[your machine].roe.ac.uk} to access your machine as if you were already on-site. For Windows and Linux computers, the OpenVPN client is recommended.
\end{enumerate}


\subsection{Accessing the ROE wifi with your own laptop}
To make use of the ROE wifi, you will need to have a VPN client and a copy of the authentication certificate installed on your laptop.  However, it is also possible to access the wifi via eduroam. 


\subsection{Printing}

Fortunately if you're using {\sc gnome} (and possibly even if you're not), making use of the networked printers on site is easier now than it has been in the past. Most applications you'll be running will have a familiar File $\rightarrow$\ Print option which will list all the networked printers on site to choose between. Printing is free for staff and PhD students, and there are several black and white laser printers in and around the PhD offices, most of which print double-sided by default. There are also a handful of colour printers to be used sensibly (i.e. not for printing novels of black text because there's a single figure on Page 254 that might look good in colour...) -- if in doubt, ask an elder PhD student where these are located.

It's also possible to print from your laptop by setting up network printing and using the IP address noted on the relevant printer, or by using a third-party solution like Google Cloud Print (which requires a Google account and that Google Chrome be installed and left running on your work machine).

\subsection{FTP}

Occasionally you'll need to make a big file or data set available to a collaborator in a different institution. With personal cloud-based storage solutions like Dropbox becoming commonplace, now might be the time to register an account, but there may well be times when you need to transfer directories larger than the 2\,GB you get free from Dropbox, and for that you'll need to make use of ROE's File Transfer Protocol (FTP) server, {\sc spider}. {\sc spider} is another stand alone Debian machine on which you will need to be given another account and allocation of disk space (helpdesk ticket!). Once this is done, access to {\sc spider} will typically only be possibly from {\sc katrine} (meaning you'll need to SSH into {\sc katrine} and from there SSH into {\sc spider}), but comprehensive (if slightly technical) instructions on how to enable access to {\sc spider} directly from your own machine are to be found at \url{http://intra.roe.ac.uk/atc/computing/docs/uun/uun13.html}. 

Data can then be transferred to your {\sc spider} directory using the standard Linux Secure CoPy (SCP) command for transfer of files across the network:

{\bf scp /file/on/your/machine spider.roe.ac.uk:/home/ftp/pub/[user name]}

and accessed by your collaborators via anonymous FTP (\textit{User name: } ``anonymous'', \textit{Password: } their email address). Feel free to ask an older student for help with setting this up.

%\subsection{USB pens}
%
%Support for USB pens and external hard drives is sporadic -- it seems to depend on how old the machine is, and what type of disk formatting exists on the USB pen. FAT32 drives tend to work fine without requiring administrator privileges to mount them, albeit are subject to the usual FAT32 constraint concerning the inability to write any individual file that's more than 3.5\,GB in size. NTFS drives and other formats \textit{can} be mounted, but generally not without admin privileges. If you do need to transfer extremely large files back and forth, then you may want to look into compressing them ({\bf tar -zcvf CompressedFile.tar.gz BigFile} to compress ``BigFile'' to ``CompressedFile.tar.gz'') and/or using the Linux ``split'' command to break a big file in to chunks of arbitrary size to get them under the 3.5\,GB limit of FAT32.
%
\subsection{stacpolly and cuillin}

Stacpolly and cuillin are the names of the IfA's mini supercomputers (or clusters).  Stacpolly is available to everyone and features many-core, large memory nodes.  
Stacpolly can be used either to do computationally or memory intensive tasks or, alternatively, to run many copies of simpler programs quickly. If you think you need
access to stacpolly, speak with your supervisor and - if they agree - submit a helpdesk ticket. Cuillin is only available to students working on particular projects.
Check with your supervisor if you qualify and - if so - email Eric Tittley (ert@roe.ac.uk).  Access to either stacpolly or cuillin then provides you with some space 
on the `head' (main) node which you can then use to compile code and farm jobs out to the worker nodes. Information about stacpolly and cuillin can be found at 
\url{http://www.roe.ac.uk/~ert/stacpolly} and \url{http://cuillin.roe.ac.uk} respectively.

%%% SANDY'S CONTRIBUTION FOLLOWS
\subsection{Mac Specific Guidance}
\label{macusershelp}
If you have already been converted to Mac by some evangelical Apple fanboy, you may find the lack of root access and apparently arcane UI design on the ROE's Linux boxes to be thoroughly upsetting.
Worry not! 
Many astronomers, particularly in the US but increasingly everywhere, now use Macs.
The following is a fairly random set of suggestions for things you might want to know/use.

{\bf{Disclaimer:}}
This is for Mac laptops, because they are now the most widely used portable astronomy platform.
It is possible to use a Windows laptop, probably with something like \url{http://x.cygwin.com/}.
This is rarely done, however, and certainly isn't the industry standard.
It is of course very possible to use a Linux-running laptop (i.e., any laptop at all).
If you're doing this, you probably know what you're doing but should read the computing notes above anyway.
If you haven't done but want to do this, you need to choose a Linux distribution. 
Google for Debian vs Ubuntu vs Suse (all different distributions) to try and help you make up your mind.
Running Ubuntu on a non-Apple laptop will let you do everything you need to without buying Apple!

\subsubsection{How do I get on the network/internet?}
See the sections on remote access and laptops above to get access. On the Mac, TunnelBlick is the OpenVPN client software you need to install. Connect to the `Staff' Wifi network and use the `all' configuration of the VPN.
If you can't get it set up, ask a fellow Mac user.

Alternatively, and probably additionally, you could use the university's VPN. On OS X Snow Leopard and lower, use the University provided Cisco VPN Client software. On Lion and above, use the OS X built in VPN software.
Full instructions are available from \url{http://www.ed.ac.uk/schools-departments/information-services/services/computing/desktop-personal/vpn}

If you haven't already, you should get set up with the eduroam network. Using this doesn't require a VPN to be set-up and can be used in a variety
of locations across the world and so is very useful when attending conferences. To login in to this you need to use the username: [your student number]@ed.ac.uk along with
the same password you would use for University VPN.


\subsubsection{How do I do any real astronomy on my Mac?}
See the wonderful Mac sections of \url{http://www.astrobetter.com/wiki/}
{\bf You should really follow the Mac Setup Guide if you haven't already used your Mac for any Astronomy}
Particularly: you may wish to install MacPorts or Fink. These are tools which let you run Unix software on your Mac.

\subsubsection{I keep hearing about X11?}
X11 is a windowing system. For your purposes, this is an app that lets you use more Linuxy software that isn't designed for Mac.
I recommend using XQuartz instead of Apple's own X11 app.
Alternatively, get XCode from the App Store (or your installation DVDs if your Mac came with it) which gets you Apple's development environment as well as X11.

\subsubsection{I really like coding in Python!}
Fantastic news! Mac has a built in Python installation but it can prove to be a bit of nightmare so instead either get a free academic license to the Enthought Python Distribution (EPD) from \url{http://www.enthought.com/products/edudownload.php} or alternatively many people use the anaconda distribution which can be found at  \url{https://store.continuum.io/cshop/anaconda/}.

Enthought includes many packages you will need to do your astronomy already and additional packages can be installed via a GUI which makes it very easy to use.

\subsubsection{I really like coding in IDL!}
Oh dear... If it is at all possible then I would recommend learning Python however if you need to use some specific IDL packages then the university has IDL licences which can go on personal laptops.
Submit a help desk ticket (be connected to the ROE network, go to \url{www.roe.ac.uk} and follow the link to the intranet) asking for it. They'll provide the IDL software and a piece of licence managing software which you need to install like any other Mac application. The licence manager can be a real pain, depending on your OS X version, root account status and whether it's your day or not.

\subsubsection{My Mac isn't fast enough!}
It may be that your provided Linux box is needed every once in a while for extra horsepower. A tool like \textsc{SExtractor} runs about four times quicker on the most up to date Linux box compared to the most up to date MacBook. Anyway, you have several options:
\begin{enumerate}
\item Set up your Linux box like everybody else and just use it as you need it alongside your Mac. In such cases, you may find tools like \url{www.dropbox.com} helpful for synchronising data. Installing dropbox on your Linux box is do-able even without root access (instructions on the website) and is of course super simple on your Mac.
\item Run everything from the Terminal. For this, you'll want to SSH into your Linux box -- see next section.
\item Do something very clever, like run a parallel python \url{http://www.parallelpython.com/} server on your Linux box and thus use its processors as if they were on your Mac, while running python code.
\end{enumerate}

\subsubsection{How do I access ROE computers from my Mac?}
If you're on the ROE's network (see above), you can just do, from the Terminal app,
{\bf ssh abr@knoydart}
to connect user abr to computer knoydart.

If you just want to transfer files, use {\bf sftp abr@knoydart} or alternatively CyberDuck if you'd rather not do it from the command line.

If you need to open a port to your Linux box, use TunnelerX.

\subsubsection{I want to write my thesis}
Use the application TeXShop to write your thesis, papers, first year reports etc in \LaTeX.

\subsubsection{I want a really nice text and code editor}
Use the application TextWrangler.

\subsubsection{I want a really nice note taking app}
Use Evernote.

\subsubsection{I want to do this version control thing that people talk about}
Use MacHG for a mercurial repository or git can be installed on your Mac from the official website, \url{http://git-scm.com/}.

\subsubsection{How do I print?}
Easy option: on your Mac, on the ROE network, go to System Preferences $\rightarrow$ Print and Scan $\rightarrow$ + $\rightarrow$ IP.
Protocol: IPP,
Address: 195.194.120.81,
then find your nearby printer (find it and look at the sticker on it for its name e.g. uni\_colour).
Queue: printers/uni\_colour
and fill in the other boxes as you wish.

Hard but maybe useful option: install Google Chrome as your browser on your Linux computer (a bit of a pain) as well as your Mac (very easy). 
Then set up Google Cloud Print. 
Combined with this app \url{http://www.webabode.com/software/cloudprint.html}, this'll let you print from anywhere on your Mac to any printer that your Linux computer knows about.
This is hard because you need to install Chrome on your Linux box, but great because you can print from anywhere in the world without being on the network.

\subsubsection{I need a FITS image viewing package}
Use DS9, because its much easier to set up that GAIA.
The DS9 FUNTOOLS might be useful to you: \url{https://www.cfa.harvard.edu/~john/funtools/}

\subsubsection{I need to back up my Mac}
Easy option: buy a 1~TB portable drive and set it up as a Time Machine on your Mac (System Preferences $\rightarrow$ Time Machine).
Hard option: use some of your allocated space on the backup server to backup your Mac using rsync over the ROE network.

\end{document}
